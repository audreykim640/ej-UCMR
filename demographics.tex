% Options for packages loaded elsewhere
\PassOptionsToPackage{unicode}{hyperref}
\PassOptionsToPackage{hyphens}{url}
\PassOptionsToPackage{dvipsnames,svgnames,x11names}{xcolor}
%
\documentclass[
  letterpaper,
  DIV=11,
  numbers=noendperiod]{scrartcl}

\usepackage{amsmath,amssymb}
\usepackage{lmodern}
\usepackage{iftex}
\ifPDFTeX
  \usepackage[T1]{fontenc}
  \usepackage[utf8]{inputenc}
  \usepackage{textcomp} % provide euro and other symbols
\else % if luatex or xetex
  \usepackage{unicode-math}
  \defaultfontfeatures{Scale=MatchLowercase}
  \defaultfontfeatures[\rmfamily]{Ligatures=TeX,Scale=1}
\fi
% Use upquote if available, for straight quotes in verbatim environments
\IfFileExists{upquote.sty}{\usepackage{upquote}}{}
\IfFileExists{microtype.sty}{% use microtype if available
  \usepackage[]{microtype}
  \UseMicrotypeSet[protrusion]{basicmath} % disable protrusion for tt fonts
}{}
\makeatletter
\@ifundefined{KOMAClassName}{% if non-KOMA class
  \IfFileExists{parskip.sty}{%
    \usepackage{parskip}
  }{% else
    \setlength{\parindent}{0pt}
    \setlength{\parskip}{6pt plus 2pt minus 1pt}}
}{% if KOMA class
  \KOMAoptions{parskip=half}}
\makeatother
\usepackage{xcolor}
\setlength{\emergencystretch}{3em} % prevent overfull lines
\setcounter{secnumdepth}{-\maxdimen} % remove section numbering
% Make \paragraph and \subparagraph free-standing
\ifx\paragraph\undefined\else
  \let\oldparagraph\paragraph
  \renewcommand{\paragraph}[1]{\oldparagraph{#1}\mbox{}}
\fi
\ifx\subparagraph\undefined\else
  \let\oldsubparagraph\subparagraph
  \renewcommand{\subparagraph}[1]{\oldsubparagraph{#1}\mbox{}}
\fi

\usepackage{color}
\usepackage{fancyvrb}
\newcommand{\VerbBar}{|}
\newcommand{\VERB}{\Verb[commandchars=\\\{\}]}
\DefineVerbatimEnvironment{Highlighting}{Verbatim}{commandchars=\\\{\}}
% Add ',fontsize=\small' for more characters per line
\usepackage{framed}
\definecolor{shadecolor}{RGB}{241,243,245}
\newenvironment{Shaded}{\begin{snugshade}}{\end{snugshade}}
\newcommand{\AlertTok}[1]{\textcolor[rgb]{0.68,0.00,0.00}{#1}}
\newcommand{\AnnotationTok}[1]{\textcolor[rgb]{0.37,0.37,0.37}{#1}}
\newcommand{\AttributeTok}[1]{\textcolor[rgb]{0.40,0.45,0.13}{#1}}
\newcommand{\BaseNTok}[1]{\textcolor[rgb]{0.68,0.00,0.00}{#1}}
\newcommand{\BuiltInTok}[1]{\textcolor[rgb]{0.00,0.23,0.31}{#1}}
\newcommand{\CharTok}[1]{\textcolor[rgb]{0.13,0.47,0.30}{#1}}
\newcommand{\CommentTok}[1]{\textcolor[rgb]{0.37,0.37,0.37}{#1}}
\newcommand{\CommentVarTok}[1]{\textcolor[rgb]{0.37,0.37,0.37}{\textit{#1}}}
\newcommand{\ConstantTok}[1]{\textcolor[rgb]{0.56,0.35,0.01}{#1}}
\newcommand{\ControlFlowTok}[1]{\textcolor[rgb]{0.00,0.23,0.31}{#1}}
\newcommand{\DataTypeTok}[1]{\textcolor[rgb]{0.68,0.00,0.00}{#1}}
\newcommand{\DecValTok}[1]{\textcolor[rgb]{0.68,0.00,0.00}{#1}}
\newcommand{\DocumentationTok}[1]{\textcolor[rgb]{0.37,0.37,0.37}{\textit{#1}}}
\newcommand{\ErrorTok}[1]{\textcolor[rgb]{0.68,0.00,0.00}{#1}}
\newcommand{\ExtensionTok}[1]{\textcolor[rgb]{0.00,0.23,0.31}{#1}}
\newcommand{\FloatTok}[1]{\textcolor[rgb]{0.68,0.00,0.00}{#1}}
\newcommand{\FunctionTok}[1]{\textcolor[rgb]{0.28,0.35,0.67}{#1}}
\newcommand{\ImportTok}[1]{\textcolor[rgb]{0.00,0.46,0.62}{#1}}
\newcommand{\InformationTok}[1]{\textcolor[rgb]{0.37,0.37,0.37}{#1}}
\newcommand{\KeywordTok}[1]{\textcolor[rgb]{0.00,0.23,0.31}{#1}}
\newcommand{\NormalTok}[1]{\textcolor[rgb]{0.00,0.23,0.31}{#1}}
\newcommand{\OperatorTok}[1]{\textcolor[rgb]{0.37,0.37,0.37}{#1}}
\newcommand{\OtherTok}[1]{\textcolor[rgb]{0.00,0.23,0.31}{#1}}
\newcommand{\PreprocessorTok}[1]{\textcolor[rgb]{0.68,0.00,0.00}{#1}}
\newcommand{\RegionMarkerTok}[1]{\textcolor[rgb]{0.00,0.23,0.31}{#1}}
\newcommand{\SpecialCharTok}[1]{\textcolor[rgb]{0.37,0.37,0.37}{#1}}
\newcommand{\SpecialStringTok}[1]{\textcolor[rgb]{0.13,0.47,0.30}{#1}}
\newcommand{\StringTok}[1]{\textcolor[rgb]{0.13,0.47,0.30}{#1}}
\newcommand{\VariableTok}[1]{\textcolor[rgb]{0.07,0.07,0.07}{#1}}
\newcommand{\VerbatimStringTok}[1]{\textcolor[rgb]{0.13,0.47,0.30}{#1}}
\newcommand{\WarningTok}[1]{\textcolor[rgb]{0.37,0.37,0.37}{\textit{#1}}}

\providecommand{\tightlist}{%
  \setlength{\itemsep}{0pt}\setlength{\parskip}{0pt}}\usepackage{longtable,booktabs,array}
\usepackage{calc} % for calculating minipage widths
% Correct order of tables after \paragraph or \subparagraph
\usepackage{etoolbox}
\makeatletter
\patchcmd\longtable{\par}{\if@noskipsec\mbox{}\fi\par}{}{}
\makeatother
% Allow footnotes in longtable head/foot
\IfFileExists{footnotehyper.sty}{\usepackage{footnotehyper}}{\usepackage{footnote}}
\makesavenoteenv{longtable}
\usepackage{graphicx}
\makeatletter
\def\maxwidth{\ifdim\Gin@nat@width>\linewidth\linewidth\else\Gin@nat@width\fi}
\def\maxheight{\ifdim\Gin@nat@height>\textheight\textheight\else\Gin@nat@height\fi}
\makeatother
% Scale images if necessary, so that they will not overflow the page
% margins by default, and it is still possible to overwrite the defaults
% using explicit options in \includegraphics[width, height, ...]{}
\setkeys{Gin}{width=\maxwidth,height=\maxheight,keepaspectratio}
% Set default figure placement to htbp
\makeatletter
\def\fps@figure{htbp}
\makeatother

\KOMAoption{captions}{tableheading}
\makeatletter
\makeatother
\makeatletter
\makeatother
\makeatletter
\@ifpackageloaded{caption}{}{\usepackage{caption}}
\AtBeginDocument{%
\ifdefined\contentsname
  \renewcommand*\contentsname{Table of contents}
\else
  \newcommand\contentsname{Table of contents}
\fi
\ifdefined\listfigurename
  \renewcommand*\listfigurename{List of Figures}
\else
  \newcommand\listfigurename{List of Figures}
\fi
\ifdefined\listtablename
  \renewcommand*\listtablename{List of Tables}
\else
  \newcommand\listtablename{List of Tables}
\fi
\ifdefined\figurename
  \renewcommand*\figurename{Figure}
\else
  \newcommand\figurename{Figure}
\fi
\ifdefined\tablename
  \renewcommand*\tablename{Table}
\else
  \newcommand\tablename{Table}
\fi
}
\@ifpackageloaded{float}{}{\usepackage{float}}
\floatstyle{ruled}
\@ifundefined{c@chapter}{\newfloat{codelisting}{h}{lop}}{\newfloat{codelisting}{h}{lop}[chapter]}
\floatname{codelisting}{Listing}
\newcommand*\listoflistings{\listof{codelisting}{List of Listings}}
\makeatother
\makeatletter
\@ifpackageloaded{caption}{}{\usepackage{caption}}
\@ifpackageloaded{subcaption}{}{\usepackage{subcaption}}
\makeatother
\makeatletter
\@ifpackageloaded{tcolorbox}{}{\usepackage[many]{tcolorbox}}
\makeatother
\makeatletter
\@ifundefined{shadecolor}{\definecolor{shadecolor}{rgb}{.97, .97, .97}}
\makeatother
\makeatletter
\makeatother
\ifLuaTeX
  \usepackage{selnolig}  % disable illegal ligatures
\fi
\IfFileExists{bookmark.sty}{\usepackage{bookmark}}{\usepackage{hyperref}}
\IfFileExists{xurl.sty}{\usepackage{xurl}}{} % add URL line breaks if available
\urlstyle{same} % disable monospaced font for URLs
\hypersetup{
  pdftitle={Appendix B},
  colorlinks=true,
  linkcolor={blue},
  filecolor={Maroon},
  citecolor={Blue},
  urlcolor={Blue},
  pdfcreator={LaTeX via pandoc}}

\title{Appendix B}
\author{}
\date{}

\begin{document}
\maketitle
\ifdefined\Shaded\renewenvironment{Shaded}{\begin{tcolorbox}[boxrule=0pt, breakable, interior hidden, enhanced, borderline west={3pt}{0pt}{shadecolor}, frame hidden, sharp corners]}{\end{tcolorbox}}\fi

\hypertarget{appendix-b}{%
\section{Appendix B}\label{appendix-b}}

All code whose results are shown in the study. (This document is just
for code transparency, not to display outputs from the code.)

\begin{Shaded}
\begin{Highlighting}[]
\FunctionTok{library}\NormalTok{(dplyr)}
\FunctionTok{library}\NormalTok{(ggplot2)}
\FunctionTok{library}\NormalTok{(knitr)}
\FunctionTok{library}\NormalTok{(kableExtra)}
\FunctionTok{library}\NormalTok{(ggfortify)}
\FunctionTok{library}\NormalTok{(performance)}
\end{Highlighting}
\end{Shaded}

\hypertarget{joining-data}{%
\subsection{Joining data}\label{joining-data}}

\begin{Shaded}
\begin{Highlighting}[]
\DocumentationTok{\#\# {-}{-}{-}{-}{-}{-}{-}SOURCING DATA{-}{-}{-}{-}{-}{-}{-}{-}}
\CommentTok{\# variables from https://api.census.gov/data/2020/acs/acs5/groups.html}
\NormalTok{bg\_shapes }\OtherTok{\textless{}{-}} \FunctionTok{get\_acs}\NormalTok{(}
  \AttributeTok{geography =} \StringTok{"block group"}\NormalTok{, }
  \AttributeTok{variables =} \FunctionTok{c}\NormalTok{(}\StringTok{"B02001\_001"}\NormalTok{, }\StringTok{"B02008\_001"}\NormalTok{, }\StringTok{"B02009\_001"}\NormalTok{, }\StringTok{"B02010\_001"}\NormalTok{, }
                \StringTok{"B02011\_001"}\NormalTok{, }\StringTok{"B19013\_001"}\NormalTok{, }\StringTok{"B03003\_001"}\NormalTok{, }\StringTok{"B03003\_003"}\NormalTok{),}
  \AttributeTok{state =} \StringTok{"CA"}\NormalTok{, }
  \AttributeTok{year =} \DecValTok{2020}\NormalTok{,}
  \AttributeTok{cache\_table =} \ConstantTok{TRUE}\NormalTok{,}
  \AttributeTok{geometry =} \ConstantTok{TRUE}
\NormalTok{) }

\CommentTok{\# CSV is too large to push {-}{-} download from https://www.epa.gov/dwucmr/occurrence{-}data{-}unregulated{-}contaminant{-}monitoring{-}rule\#4}
\NormalTok{UCMR }\OtherTok{\textless{}{-}} \FunctionTok{read.csv}\NormalTok{(}\StringTok{"data/UCMR4\_All.csv"}\NormalTok{) }\SpecialCharTok{\%\textgreater{}\%} 
  \FunctionTok{filter}\NormalTok{(PWSName }\SpecialCharTok{\%in\%} \FunctionTok{c}\NormalTok{(}\StringTok{"Pala North"}\NormalTok{, }\StringTok{"Pauma"}\NormalTok{, }\StringTok{"Pechanga"}\NormalTok{, }\StringTok{"Redding Rancheria"}\NormalTok{, }
                        \StringTok{"Viejas Community System"}\NormalTok{) }\SpecialCharTok{|}\NormalTok{ State }\SpecialCharTok{==} \StringTok{"CA"}\NormalTok{) }

\NormalTok{govwatershapes }\OtherTok{\textless{}{-}} \FunctionTok{st\_read}\NormalTok{(}\StringTok{"data/SABL\_Public\_230403"}\NormalTok{) }\SpecialCharTok{\%\textgreater{}\%} 
  \FunctionTok{rename}\NormalTok{(}\StringTok{"PWSID"} \OtherTok{=} \StringTok{"SABL\_PWSID"}\NormalTok{) }\SpecialCharTok{\%\textgreater{}\%}
  \FunctionTok{filter}\NormalTok{(PWSID }\SpecialCharTok{\%in\%} \FunctionTok{unique}\NormalTok{(UCMR}\SpecialCharTok{$}\NormalTok{PWSID)) }\SpecialCharTok{\%\textgreater{}\%} 
  \FunctionTok{st\_transform}\NormalTok{(}\DecValTok{4269}\NormalTok{)}

\CommentTok{\# .gpkg is too large to push {-}{-} download from https://github.com/SimpleLab{-}Inc/wsb}
\NormalTok{SLwatershapes }\OtherTok{\textless{}{-}} \FunctionTok{st\_read}\NormalTok{(}\StringTok{"data/temm.gpkg"}\NormalTok{) }\SpecialCharTok{\%\textgreater{}\%} 
  \CommentTok{\# https://community.rstudio.com/t/how{-}to{-}open{-}a{-}geopackage{-}file/41027}
  \FunctionTok{filter}\NormalTok{(pwsid }\SpecialCharTok{\%in\%} \FunctionTok{unique}\NormalTok{(UCMR}\SpecialCharTok{$}\NormalTok{PWSID))}


\DocumentationTok{\#\# {-}{-}{-}{-}{-}{-}{-}COMBINING CA PWS AND SIMPLELAB GEOMETRIES{-}{-}{-}{-}{-}{-}{-}{-}}
\NormalTok{missingPWSs }\OtherTok{\textless{}{-}} \FunctionTok{setdiff}\NormalTok{(}\FunctionTok{unique}\NormalTok{(UCMR}\SpecialCharTok{$}\NormalTok{PWSID), }\FunctionTok{unique}\NormalTok{(govwatershapes}\SpecialCharTok{$}\NormalTok{PWSID)) }\SpecialCharTok{\%\textgreater{}\%} 
  \FunctionTok{map}\NormalTok{(}\ControlFlowTok{function}\NormalTok{ (x) \{}\FunctionTok{ifelse}\NormalTok{(}\FunctionTok{substr}\NormalTok{(x,}\DecValTok{1}\NormalTok{,}\DecValTok{1}\NormalTok{) }\SpecialCharTok{==} \StringTok{"9"}\NormalTok{, }\FunctionTok{paste0}\NormalTok{(}\StringTok{"0"}\NormalTok{, x), x)\}) }

\NormalTok{missingshapes }\OtherTok{\textless{}{-}}\NormalTok{ SLwatershapes }\SpecialCharTok{\%\textgreater{}\%} 
  \FunctionTok{filter}\NormalTok{(pwsid }\SpecialCharTok{\%in\%}\NormalTok{ missingPWSs) }\SpecialCharTok{\%\textgreater{}\%} 
  \FunctionTok{rename}\NormalTok{(}\StringTok{"PWSID"} \OtherTok{=} \StringTok{"pwsid"}\NormalTok{,}
         \StringTok{"WATER\_SY\_1"} \OtherTok{=} \StringTok{"pws\_name"}\NormalTok{,}
         \StringTok{"geometry"} \OtherTok{=} \StringTok{"geom"}\NormalTok{) }\SpecialCharTok{\%\textgreater{}\%} 
  \FunctionTok{select}\NormalTok{(PWSID, WATER\_SY\_1, geometry) }\SpecialCharTok{\%\textgreater{}\%} 
  \FunctionTok{st\_transform}\NormalTok{(}\DecValTok{4269}\NormalTok{)}

\NormalTok{PWS }\OtherTok{\textless{}{-}} \FunctionTok{bind\_rows}\NormalTok{(missingshapes, govwatershapes)}

\DocumentationTok{\#\# ACS variable codes }
\CommentTok{\# B02001\_001 Race total B02001\_002 White alone\textbackslash{}}
\CommentTok{\# B02009\_001 Black alone or in combination }
\CommentTok{\# B02010\_001 Native American alone or in combination }
\CommentTok{\# B02011\_001 Asian alone or in combination }
\CommentTok{\# Hispanic total: B03003\_001 }
\CommentTok{\# B03003\_003 yes hispanic or latino}

\DocumentationTok{\#\# {-}{-}{-}{-}{-}{-}{-}TRANSFORMING DEMOGRAPHICS DATA{-}{-}{-}{-}{-}{-}{-}{-}}
\NormalTok{bg\_shapes\_wide }\OtherTok{\textless{}{-}}\NormalTok{ bg\_shapes }\SpecialCharTok{\%\textgreater{}\%} 
  \FunctionTok{select}\NormalTok{(}\SpecialCharTok{{-}}\NormalTok{moe) }\SpecialCharTok{\%\textgreater{}\%} 
  \FunctionTok{pivot\_wider}\NormalTok{(}\AttributeTok{names\_from =} \StringTok{"variable"}\NormalTok{, }\AttributeTok{values\_from =} \StringTok{"estimate"}\NormalTok{) }\SpecialCharTok{\%\textgreater{}\%} 
  \FunctionTok{rename}\NormalTok{(}\AttributeTok{Total =}\NormalTok{ B02001\_001,    }
         \AttributeTok{White =}\NormalTok{ B02008\_001,    }
         \AttributeTok{Black =}\NormalTok{ B02009\_001,}
         \AttributeTok{Indigenous =}\NormalTok{ B02010\_001,   }
         \AttributeTok{Asian =}\NormalTok{ B02011\_001,}
         \AttributeTok{Total\_Hisp =}\NormalTok{ B03003\_001,   }
         \AttributeTok{Hisp\_Lat =}\NormalTok{ B03003\_003,}
         \AttributeTok{Med\_Income =}\NormalTok{ B19013\_001) }\CommentTok{\#\%\textgreater{}\% }
\CommentTok{\# mutate(White = White / Total,}
\CommentTok{\#        Black = Black / Total,}
\CommentTok{\#        Indigenous = Indigenous / Total,}
\CommentTok{\#        Asian = Asian / Total,}
\CommentTok{\#        Hisp\_Lat = Hisp\_Lat / Total\_Hisp}
\CommentTok{\#        )}

\CommentTok{\# Save until after having joined since weighted averages won\textquotesingle{}t work with props}
\CommentTok{\# Want to use counts}

\CommentTok{\# block\_shapes\_tall\_props \textless{}{-} block\_shapes\_wide \%\textgreater{}\% }
\CommentTok{\#   pivot\_longer(c(White, Black, Asian, Indigenous, Hisp\_Lat), }
\CommentTok{\#                names\_to = "race", values\_to = "prop")}
\CommentTok{\# }
\CommentTok{\# ggplot(block\_shapes\_tall2, aes(x = race, y = prop, color = race)) + geom\_col()}
\CommentTok{\# }
\CommentTok{\# block\_shapes\_vals \textless{}{-} block\_shapes \%\textgreater{}\% }
\CommentTok{\#   filter(!(Total == 0 \& Total\_Hisp == 0))}

\DocumentationTok{\#\# {-}{-}{-}{-}{-}{-}{-}JOINING AND AVERAGING DEMOGRAPHICS{-}{-}{-}{-}{-}{-}{-}{-}}
\CommentTok{\# below line sourced from https://github.com/r{-}spatial/sf/issues/493}
\FunctionTok{sf\_use\_s2}\NormalTok{(}\ConstantTok{FALSE}\NormalTok{) }\CommentTok{\# flattens shapes}

\CommentTok{\# join \textless{}{-} watershapes \%\textgreater{}\% }
\CommentTok{\#   inner\_join(UCMR\_CA, by = c("pwsid" = "PWSID")) \%\textgreater{}\% }
\CommentTok{\#   select(pwsid, PWSName, primacy\_agency\_code, primary\_source\_code, tier, FacilityWaterType, CollectionDate, Contaminant, MRL, AnalyticalResultsSign, AnalyticalResultValue.µg.L.) \%\textgreater{}\% }
\CommentTok{\#   st\_intersection(bg\_shapes) \%\textgreater{}\% }
\CommentTok{\#   mutate(pwi\_county = paste0(pwsid, "{-}", GEOID))}


\CommentTok{\# PWS\_demos\_joined \textless{}{-} st\_join(govwatershapes, bg\_shapes\_wide)}

\NormalTok{shape\_intersections }\OtherTok{\textless{}{-}} \FunctionTok{st\_intersection}\NormalTok{(PWS, bg\_shapes\_wide) }\SpecialCharTok{\%\textgreater{}\%} 
  \FunctionTok{mutate}\NormalTok{(}\AttributeTok{PWS\_bg =} \FunctionTok{paste0}\NormalTok{(PWSID, }\StringTok{"{-}"}\NormalTok{, GEOID))}

\NormalTok{PWS\_aw\_demographics }\OtherTok{\textless{}{-}}\NormalTok{ shape\_intersections }\SpecialCharTok{\%\textgreater{}\%} 
  \CommentTok{\#{-}{-}{-} get area {-}{-}{-}\#}
  \FunctionTok{mutate}\NormalTok{(}\AttributeTok{area =} \FunctionTok{as.numeric}\NormalTok{(}\FunctionTok{st\_area}\NormalTok{(.))) }\SpecialCharTok{\%\textgreater{}\%} 
  \CommentTok{\#{-}{-}{-} get area{-}weight by HUC unit {-}{-}{-}\#}
  \FunctionTok{group\_by}\NormalTok{(PWSID) }\SpecialCharTok{\%\textgreater{}\%} 
  \FunctionTok{mutate}\NormalTok{(}\AttributeTok{weight =}\NormalTok{ area }\SpecialCharTok{/} \FunctionTok{sum}\NormalTok{(area)) }\SpecialCharTok{\%\textgreater{}\%} 
  \CommentTok{\#{-}{-}{-} calculate area{-}weighted corn acreage by HUC unit {-}{-}{-}\#}
  \CommentTok{\# \textquotesingle{}aw\textquotesingle{} stands for area{-}weighted }
  \FunctionTok{summarize}\NormalTok{(}\AttributeTok{Total\_aw =} \FunctionTok{sum}\NormalTok{(weight }\SpecialCharTok{*}\NormalTok{ Total),}
            \AttributeTok{White\_aw =} \FunctionTok{sum}\NormalTok{(weight }\SpecialCharTok{*}\NormalTok{ White),}
            \AttributeTok{Black\_aw =} \FunctionTok{sum}\NormalTok{(weight }\SpecialCharTok{*}\NormalTok{ Black),}
            \AttributeTok{Indig\_aw =} \FunctionTok{sum}\NormalTok{(weight }\SpecialCharTok{*}\NormalTok{ Indigenous),}
            \AttributeTok{Asian\_aw =} \FunctionTok{sum}\NormalTok{(weight }\SpecialCharTok{*}\NormalTok{ Asian),}
            \AttributeTok{Total\_HispLat\_aw =} \FunctionTok{sum}\NormalTok{(weight }\SpecialCharTok{*}\NormalTok{ Total\_Hisp),}
            \AttributeTok{HispLat\_aw =} \FunctionTok{sum}\NormalTok{(weight }\SpecialCharTok{*}\NormalTok{ Hisp\_Lat),}
            \AttributeTok{MedIncome\_aw =} \FunctionTok{sum}\NormalTok{(weight }\SpecialCharTok{*}\NormalTok{ Med\_Income)}
\NormalTok{  )}

\CommentTok{\# saving for posterity (to not have to rerun code)}
\CommentTok{\# st\_write(PWS\_aw\_demographics, dsn = \textquotesingle{}data/joinedPWSdemos/joinedPWSdemos.shp\textquotesingle{})}

\DocumentationTok{\#\# {-}{-}{-}{-}{-}{-}{-}WRANGLING DEMOGRAPHICS{-}{-}{-}{-}{-}{-}{-}{-}}
\NormalTok{PWSdemog }\OtherTok{\textless{}{-}}\NormalTok{ PWS\_aw\_demographics }\SpecialCharTok{\%\textgreater{}\%} 
  \FunctionTok{mutate}\NormalTok{(}\AttributeTok{White\_prop =}\NormalTok{ White\_aw }\SpecialCharTok{/}\NormalTok{ Total\_aw,}
         \AttributeTok{Black\_prop =}\NormalTok{ Black\_aw }\SpecialCharTok{/}\NormalTok{ Total\_aw,}
         \AttributeTok{Indig\_prop =}\NormalTok{ Indig\_aw }\SpecialCharTok{/}\NormalTok{ Total\_aw,}
         \AttributeTok{Asian\_prop =}\NormalTok{ Asian\_aw }\SpecialCharTok{/}\NormalTok{ Total\_aw,}
         \AttributeTok{HispLat\_prop =}\NormalTok{ HispLat\_aw }\SpecialCharTok{/}\NormalTok{ Total\_HispLat\_aw)}

\NormalTok{PWSdemog }\OtherTok{\textless{}{-}}\NormalTok{ PWSdemog }\SpecialCharTok{\%\textgreater{}\%} 
  \FunctionTok{st\_drop\_geometry}\NormalTok{() }\SpecialCharTok{\%\textgreater{}\%} \CommentTok{\# max.col() couldn\textquotesingle{}t work with }
  \FunctionTok{mutate}\NormalTok{(}\AttributeTok{Predom =} \FunctionTok{colnames}\NormalTok{(.[}\DecValTok{10}\SpecialCharTok{:}\DecValTok{13}\NormalTok{])[}\FunctionTok{max.col}\NormalTok{(.[}\DecValTok{10}\SpecialCharTok{:}\DecValTok{13}\NormalTok{])]) }\SpecialCharTok{\%\textgreater{}\%} 
  \CommentTok{\# above line courtesy of https://stackoverflow.com/questions/17735859/for{-}each{-}row{-}return{-}the{-}column{-}name{-}of{-}the{-}largest{-}value}
  \FunctionTok{right\_join}\NormalTok{(PWSdemog) }\SpecialCharTok{\%\textgreater{}\%} 
  \FunctionTok{select}\NormalTok{(PWSID, MedIncome\_aw, White\_prop, Black\_prop,}
\NormalTok{         Indig\_prop, Asian\_prop, HispLat\_prop, Predom)}

\DocumentationTok{\#\# {-}{-}{-}{-}{-}{-}{-}JOINING W UCMR{-}{-}{-}{-}{-}{-}{-}{-}}
\NormalTok{contamsOfInterest }\OtherTok{\textless{}{-}} \FunctionTok{c}\NormalTok{(}\StringTok{"germanium"}\NormalTok{, }\StringTok{"manganese"}\NormalTok{, }\StringTok{"HAA6Br"}\NormalTok{, }\StringTok{"HAA9"}\NormalTok{)}

\NormalTok{full }\OtherTok{\textless{}{-}}\NormalTok{ UCMR }\SpecialCharTok{\%\textgreater{}\%} 
  \FunctionTok{select}\NormalTok{(PWSID, PWSName, Contaminant, AnalyticalResultsSign,}
\NormalTok{         AnalyticalResultValue.µg.L.) }\SpecialCharTok{\%\textgreater{}\%} 
  \FunctionTok{filter}\NormalTok{(Contaminant }\SpecialCharTok{\%in\%}\NormalTok{ contamsOfInterest) }\SpecialCharTok{\%\textgreater{}\%} 
  \FunctionTok{inner\_join}\NormalTok{(PWSdemog) }\SpecialCharTok{\%\textgreater{}\%} 
  \FunctionTok{pivot\_longer}\NormalTok{(}\FunctionTok{c}\NormalTok{(White\_prop, Black\_prop, Asian\_prop, Indig\_prop), }
               \AttributeTok{names\_to =} \StringTok{"RaceEth"}\NormalTok{, }\AttributeTok{values\_to =} \StringTok{"Prop"}\NormalTok{) }\SpecialCharTok{\%\textgreater{}\%} 
  \FunctionTok{rename}\NormalTok{(}\AttributeTok{ContamLevel =}\NormalTok{ AnalyticalResultValue.µg.L.,}
         \AttributeTok{RelativetoMRL =}\NormalTok{ AnalyticalResultsSign,}
         \AttributeTok{MedIncome =}\NormalTok{ MedIncome\_aw) }\SpecialCharTok{\%\textgreater{}\%} 
  \FunctionTok{filter}\NormalTok{(ContamLevel }\SpecialCharTok{!=} \DecValTok{0}\NormalTok{) }\SpecialCharTok{\%\textgreater{}\%} 
  \FunctionTok{mutate}\NormalTok{(}\AttributeTok{Predom =} \FunctionTok{substr}\NormalTok{(Predom, }\DecValTok{1}\NormalTok{, }\FunctionTok{nchar}\NormalTok{(Predom)}\SpecialCharTok{{-}}\DecValTok{5}\NormalTok{),}
         \AttributeTok{RaceEth =} \FunctionTok{substr}\NormalTok{(RaceEth, }\DecValTok{1}\NormalTok{, }\FunctionTok{nchar}\NormalTok{(RaceEth)}\SpecialCharTok{{-}}\DecValTok{5}\NormalTok{),}
         \AttributeTok{RaceEth =} \FunctionTok{ifelse}\NormalTok{(RaceEth }\SpecialCharTok{==} \StringTok{"Indig"}\NormalTok{, }\StringTok{"Indigenous"}\NormalTok{, RaceEth),}
         \AttributeTok{RaceEth =} \FunctionTok{ifelse}\NormalTok{(RaceEth }\SpecialCharTok{==} \StringTok{"White"}\NormalTok{, }\StringTok{"white"}\NormalTok{, RaceEth),}
         \AttributeTok{RaceEth =} \FunctionTok{factor}\NormalTok{(RaceEth),}
         \AttributeTok{Contaminant =} \FunctionTok{ordered}\NormalTok{(Contaminant, }
                              \AttributeTok{levels =} \FunctionTok{c}\NormalTok{(}\StringTok{"germanium"}\NormalTok{,}\StringTok{"manganese"}\NormalTok{, }\StringTok{"HAA6Br"}\NormalTok{, }\StringTok{"HAA9"}\NormalTok{)),}
         \AttributeTok{LogLevel =} \FunctionTok{log10}\NormalTok{(ContamLevel),}
         \AttributeTok{HLFiftyPer =} \FunctionTok{ifelse}\NormalTok{(HispLat\_prop }\SpecialCharTok{\textgreater{}=} \FunctionTok{median}\NormalTok{(HispLat\_prop), }\StringTok{"Above"}\NormalTok{, }\StringTok{"Below"}\NormalTok{),}
         \AttributeTok{HLFiftyPer =} \FunctionTok{factor}\NormalTok{(HLFiftyPer, }\AttributeTok{levels =} \FunctionTok{c}\NormalTok{(}\StringTok{"Above"}\NormalTok{, }\StringTok{"Below"}\NormalTok{))) }

\FunctionTok{write.csv}\NormalTok{(full, }\StringTok{"full.csv"}\NormalTok{, }\AttributeTok{row.names=}\ConstantTok{FALSE}\NormalTok{)}
\end{Highlighting}
\end{Shaded}

Description of data joining: I began by filtering the public water
system boundary data based on what was available in the UCMR data to
pare down the spatial data to only what will be relevant for the
analysis. Then, I joined it with the spatial demographics data from the
American Community Survey. This join was via spatial joins, which are
computationally expensive and slow. (Joining PWS and UCMR data would
create a larger dataset, which would be even slower, so I started here.)
Using the sf package, I found which block groups overlapped each PWS.
Then for each PWS, I attributed a proportion (or weight) to each block
group, representing how much of the PWS boundary it intersected with.
For instance, if part of block group 1 covered a quarter of the PWS in
question, block group 1 received a 0.25. Then, other block groups
overlapping with the PWS are assigned corresponding weights for that
PWS. It is feasible that block groups appear multiple times in the
dataset, receiving weights for different PWSs that they overlap with.
After finding the weights, the weight for block group 1's overlap with
the PWS in questions is multiplied by the race, ethnicity, and total
respondent counts. This statistic by weight multiplication occurs for
each PWS-block group intersection, with PWSs possibly having multiple
entries if it overlaps with multiple block groups. Then, for each PWS,
all of the weighted statistics for that PWS are added together to create
a cumulative weighted average for each of the statistics. The result of
this join is a dataframe with each PWS appearing once with spatial data
and weighted average counts of Asian, Black, Indigenous, and white
respondents as well as a weighted average total of respondents to the
race question. Similarly, the data contains a weighted average count of
Hispanic/Latin(o/a) respondents and a weighted average total count of
respondents to the Hispanic/Latin(o/a) question. These weighted averages
are often rough estimates, as they make the assumption that the
distribution of counts is even and consistent throughout an area whereas
in reality, it's possible that people are not evenly distributed within
an area. For instance, it's possible that there exist no people in one
particular overlapping part of a block group and PWS but the counts from
that block group will still be weighted and attributed to that PWS. With
this joined dataset, I performed some further data wrangling for later
data analysis, including converting race/ethnicity counts to proportions
to create more values that are more comparable across PWSs and creating
a categorical variable that lists the race that makes up the greatest
proportion of the population that a PWS serves (according to this data).
I additionally siphoned ethnicity into a categorical variable indicating
if a public water system is ``Above'' or ``Below'' the 50th percentile
for proportion of Hispanic or Latino/a constituents. Then I joined this
PWS-demographics dataset to the UCMR data using the dplyr package (like
most of this data wrangling work).

\hypertarget{race-tests}{%
\subsection{Race tests}\label{race-tests}}

\hypertarget{racial-groupings-regressions}{%
\subsubsection{Racial groupings
regressions}\label{racial-groupings-regressions}}

In the following analyses, I am mainly testing for the following
hypotheses:

\begin{itemize}
\tightlist
\item
  Null hypothesis: The relationship between the amount of germanium in
  the water and proportion of race will NOT depend on (change with) the
  race.
\item
  Alternative hypothesis: The relationship between the amount of
  germanium in the water and proportion of race WILL depend on (change
  with) the race.
\end{itemize}

The basic form of the output is

where \(\alpha\) and the \(\beta\)s are the values found from the
regression, and each \(\operatorname{RaceEth}_{\operatorname{x}}\)
serves as an ``on/off'' switch where if the RaceEth variable is ``x''
(e.g., Black) at that time, the
\(\operatorname{RaceEth}_{\operatorname{Black}}\) will equal 1 and all
others will equal 0, so the coefficients corresponding with Black will
have an impact while the other coefficients will be ``turned off''.

\hypertarget{germanium-levels-vs-race-served-by-pws-proportions}{%
\paragraph{Germanium levels vs Race served by PWS
proportions}\label{germanium-levels-vs-race-served-by-pws-proportions}}

\begin{Shaded}
\begin{Highlighting}[]
\NormalTok{germanium }\OtherTok{\textless{}{-}}\NormalTok{ full }\SpecialCharTok{\%\textgreater{}\%} 
  \FunctionTok{filter}\NormalTok{(Contaminant }\SpecialCharTok{==} \StringTok{"germanium"}\NormalTok{)}

\FunctionTok{ggplot}\NormalTok{(germanium, }\FunctionTok{aes}\NormalTok{(}\AttributeTok{x =}\NormalTok{ Prop, }\AttributeTok{y =}\NormalTok{ LogLevel)) }\SpecialCharTok{+}
  \FunctionTok{geom\_point}\NormalTok{(}\AttributeTok{alpha =} \DecValTok{1}\NormalTok{) }\SpecialCharTok{+}
  \FunctionTok{geom\_smooth}\NormalTok{(}\AttributeTok{se =} \ConstantTok{FALSE}\NormalTok{, }\AttributeTok{method =} \StringTok{"lm"}\NormalTok{, }\AttributeTok{color =} \StringTok{"goldenrod"}\NormalTok{) }\SpecialCharTok{+}
  \FunctionTok{facet\_grid}\NormalTok{(}\SpecialCharTok{\textasciitilde{}}\NormalTok{ RaceEth, }\AttributeTok{scale =} \StringTok{"free"}\NormalTok{) }\SpecialCharTok{+}
  \FunctionTok{theme\_classic}\NormalTok{() }\SpecialCharTok{+}
  \FunctionTok{scale\_x\_continuous}\NormalTok{(}\AttributeTok{labels =}\NormalTok{ scales}\SpecialCharTok{::}\NormalTok{percent)}

\NormalTok{germanium\_model }\OtherTok{\textless{}{-}} \FunctionTok{glm}\NormalTok{(LogLevel }\SpecialCharTok{\textasciitilde{}}\NormalTok{ Prop }\SpecialCharTok{*}\NormalTok{ RaceEth, }\AttributeTok{data =}\NormalTok{ germanium)}
\FunctionTok{summary}\NormalTok{(germanium\_model)}
\end{Highlighting}
\end{Shaded}

Since the response variable here is the log (base-10) of the contaminant
concentration, the coefficients are most interpretable when placed as
the coefficent of 10 to undo the transformation. The baseline in this
model is the proportion of Asian respondents, which observes a
staistically significant (p-value \(\approx\) 0) negative correlation
with the level of germanium. According to this model for each each
percent increase of Asian respondents, we would expect to observe, on
average, the concentration of germanium in the water decrease by
approximately 0.040 \(\mu\)g/L. While the p-value is negligible, we
cannot say that it equals zerol it is only very close.

For germanium, there appears to be sufficient evidence to reject the
null hypothesis that the race does not have an impact on the
relationship between proportion of race and concentration of the
contaminant. Both white and Hispanic/Latino populations experience an
increase in germanium concentration in water as the proportion of those
races increase.

For each percent increase in the proportion of people that the water
system being Hispanic/Latino, we would expect to observe, on average,
the concentration of germanium in the water \emph{increase} by (-4.016 +
5.688 \(\approx\) ) 0.017 \(\mu\)g/L. The rate of increase becomes
(-4.016 + 5.999 \(\approx\) ) 0.020 \(\mu\)g/L for each percent increase
in the proportion of served people being white.

The relationship between the proportions of Black and Indigenous
populations and contaminant levels visually seems to observe a similar
pattern demonstrated by Asian populations. Statistically, there is no
evidence that there is difference in either slope or intercept for these
three groups.

\hypertarget{manganese-levels-vs-race-served-by-pws-proportions}{%
\paragraph{Manganese levels vs Race served by PWS
proportions}\label{manganese-levels-vs-race-served-by-pws-proportions}}

\begin{Shaded}
\begin{Highlighting}[]
\NormalTok{manganese }\OtherTok{\textless{}{-}}\NormalTok{ full }\SpecialCharTok{\%\textgreater{}\%} 
  \FunctionTok{filter}\NormalTok{(Contaminant }\SpecialCharTok{==} \StringTok{"manganese"}\NormalTok{)}

\FunctionTok{ggplot}\NormalTok{(manganese, }\FunctionTok{aes}\NormalTok{(}\AttributeTok{x =}\NormalTok{ Prop, }\AttributeTok{y =}\NormalTok{ LogLevel)) }\SpecialCharTok{+}
  \FunctionTok{geom\_point}\NormalTok{(}\AttributeTok{alpha =} \DecValTok{1}\NormalTok{) }\SpecialCharTok{+}
  \FunctionTok{geom\_smooth}\NormalTok{(}\AttributeTok{se =} \ConstantTok{FALSE}\NormalTok{, }\AttributeTok{method =} \StringTok{"lm"}\NormalTok{, }\AttributeTok{color =} \StringTok{"goldenrod"}\NormalTok{) }\SpecialCharTok{+}
  \FunctionTok{facet\_grid}\NormalTok{(}\SpecialCharTok{\textasciitilde{}}\NormalTok{ RaceEth, }\AttributeTok{switch =} \StringTok{"both"}\NormalTok{, }\AttributeTok{scale =} \StringTok{"free"}\NormalTok{) }\SpecialCharTok{+}
  \FunctionTok{theme\_classic}\NormalTok{() }\SpecialCharTok{+}
  \FunctionTok{scale\_x\_continuous}\NormalTok{(}\AttributeTok{labels =}\NormalTok{ scales}\SpecialCharTok{::}\NormalTok{percent)}

\NormalTok{manganese\_model }\OtherTok{\textless{}{-}} \FunctionTok{glm}\NormalTok{(LogLevel }\SpecialCharTok{\textasciitilde{}}\NormalTok{ Prop }\SpecialCharTok{*}\NormalTok{ RaceEth, }\AttributeTok{data =}\NormalTok{ manganese)}
\FunctionTok{summary}\NormalTok{(manganese\_model)}
\end{Highlighting}
\end{Shaded}

\hypertarget{haa6br-levels-vs-race-served-by-pws-proportions}{%
\paragraph{HAA6Br levels vs Race served by PWS
proportions}\label{haa6br-levels-vs-race-served-by-pws-proportions}}

\begin{Shaded}
\begin{Highlighting}[]
\NormalTok{HAA6Br }\OtherTok{\textless{}{-}}\NormalTok{ full }\SpecialCharTok{\%\textgreater{}\%} 
  \FunctionTok{filter}\NormalTok{(Contaminant }\SpecialCharTok{==} \StringTok{"HAA6Br"}\NormalTok{)}

\FunctionTok{ggplot}\NormalTok{(HAA6Br, }\FunctionTok{aes}\NormalTok{(}\AttributeTok{x =}\NormalTok{ Prop, }\AttributeTok{y =}\NormalTok{ LogLevel)) }\SpecialCharTok{+}
  \FunctionTok{geom\_point}\NormalTok{(}\AttributeTok{alpha =} \DecValTok{1}\NormalTok{) }\SpecialCharTok{+}
  \FunctionTok{geom\_smooth}\NormalTok{(}\AttributeTok{se =} \ConstantTok{FALSE}\NormalTok{, }\AttributeTok{method =} \StringTok{"lm"}\NormalTok{, }\AttributeTok{color =} \StringTok{"goldenrod"}\NormalTok{) }\SpecialCharTok{+}
  \FunctionTok{facet\_grid}\NormalTok{(}\SpecialCharTok{\textasciitilde{}}\NormalTok{ RaceEth, }\AttributeTok{scale =} \StringTok{"free"}\NormalTok{) }\SpecialCharTok{+}
  \FunctionTok{theme\_classic}\NormalTok{() }\SpecialCharTok{+}
  \FunctionTok{scale\_x\_continuous}\NormalTok{(}\AttributeTok{labels =}\NormalTok{ scales}\SpecialCharTok{::}\NormalTok{percent)}

\NormalTok{HAA6Br\_model }\OtherTok{\textless{}{-}} \FunctionTok{glm}\NormalTok{(LogLevel }\SpecialCharTok{\textasciitilde{}}\NormalTok{ Prop }\SpecialCharTok{*}\NormalTok{ RaceEth, }\AttributeTok{data =}\NormalTok{ HAA6Br)}
\FunctionTok{summary}\NormalTok{(HAA6Br\_model)}
\end{Highlighting}
\end{Shaded}

\hypertarget{haa9-levels-vs-race-served-by-pws-proportions}{%
\paragraph{HAA9 levels vs Race served by PWS
proportions}\label{haa9-levels-vs-race-served-by-pws-proportions}}

\begin{Shaded}
\begin{Highlighting}[]
\NormalTok{HAA9 }\OtherTok{\textless{}{-}}\NormalTok{ full }\SpecialCharTok{\%\textgreater{}\%} 
  \FunctionTok{filter}\NormalTok{(Contaminant }\SpecialCharTok{==} \StringTok{"HAA9"}\NormalTok{)}
  
\FunctionTok{ggplot}\NormalTok{(HAA9, }\FunctionTok{aes}\NormalTok{(}\AttributeTok{x =}\NormalTok{ Prop, }\AttributeTok{y =}\NormalTok{ LogLevel)) }\SpecialCharTok{+}
  \FunctionTok{geom\_point}\NormalTok{(}\AttributeTok{alpha =} \DecValTok{1}\NormalTok{) }\SpecialCharTok{+}
  \FunctionTok{geom\_smooth}\NormalTok{(}\AttributeTok{se =} \ConstantTok{FALSE}\NormalTok{, }\AttributeTok{method =} \StringTok{"lm"}\NormalTok{, }\AttributeTok{color =} \StringTok{"goldenrod"}\NormalTok{) }\SpecialCharTok{+}
  \FunctionTok{facet\_grid}\NormalTok{(}\SpecialCharTok{\textasciitilde{}}\NormalTok{ RaceEth, }\AttributeTok{switch =} \StringTok{"both"}\NormalTok{, }\AttributeTok{scale =} \StringTok{"free"}\NormalTok{) }\SpecialCharTok{+}
  \FunctionTok{theme\_classic}\NormalTok{() }\SpecialCharTok{+}
  \FunctionTok{scale\_x\_continuous}\NormalTok{(}\AttributeTok{labels =}\NormalTok{ scales}\SpecialCharTok{::}\NormalTok{percent)}

\NormalTok{HAA9\_model }\OtherTok{\textless{}{-}} \FunctionTok{glm}\NormalTok{(LogLevel }\SpecialCharTok{\textasciitilde{}}\NormalTok{ Prop }\SpecialCharTok{*}\NormalTok{ RaceEth, }\AttributeTok{data =}\NormalTok{ HAA9)}
\FunctionTok{summary}\NormalTok{(HAA9\_model)}
\end{Highlighting}
\end{Shaded}

\hypertarget{race-kruskal-wallis-tests}{%
\subsubsection{Race Kruskal-Wallis
tests}\label{race-kruskal-wallis-tests}}

\begin{Shaded}
\begin{Highlighting}[]
\FunctionTok{ggplot}\NormalTok{(full, }\FunctionTok{aes}\NormalTok{(}\AttributeTok{x =}\NormalTok{ Predom, }\AttributeTok{y =}\NormalTok{ ContamLevel)) }\SpecialCharTok{+}
  \FunctionTok{geom\_violin}\NormalTok{(}\AttributeTok{color =} \StringTok{"black"}\NormalTok{, }\AttributeTok{fill =} \StringTok{"goldenrod"}\NormalTok{) }\SpecialCharTok{+}
  \FunctionTok{scale\_y\_log10}\NormalTok{() }\SpecialCharTok{+}
  \FunctionTok{facet\_wrap}\NormalTok{(}\SpecialCharTok{\textasciitilde{}}\NormalTok{ Contaminant) }\SpecialCharTok{+}
  \FunctionTok{theme\_classic}\NormalTok{() }\SpecialCharTok{+}
  \FunctionTok{labs}\NormalTok{(}\AttributeTok{title =} \StringTok{"Contaminant concentrations by predominant race"}\NormalTok{,}
       \AttributeTok{subtitle =} \StringTok{"Note: y{-}axis is log{-}adjusted"}\NormalTok{,}
       \AttributeTok{x =} \StringTok{"Predominant racial group"}\NormalTok{,}
       \AttributeTok{y =} \StringTok{"Contaminant level (\textbackslash{}u03bcg/L)"}\NormalTok{)}
\end{Highlighting}
\end{Shaded}

\begin{Shaded}
\begin{Highlighting}[]
\FunctionTok{kruskal.test}\NormalTok{(ContamLevel }\SpecialCharTok{\textasciitilde{}}\NormalTok{ Predom, }\AttributeTok{data =}\NormalTok{ manganese }\SpecialCharTok{\%\textgreater{}\%} \FunctionTok{group\_by}\NormalTok{(PWSID) }\SpecialCharTok{\%\textgreater{}\%} \FunctionTok{slice}\NormalTok{(}\DecValTok{1}\NormalTok{))}
\FunctionTok{kruskal.test}\NormalTok{(ContamLevel }\SpecialCharTok{\textasciitilde{}}\NormalTok{ Predom, }\AttributeTok{data =}\NormalTok{ germanium }\SpecialCharTok{\%\textgreater{}\%} \FunctionTok{group\_by}\NormalTok{(PWSID) }\SpecialCharTok{\%\textgreater{}\%} \FunctionTok{slice}\NormalTok{(}\DecValTok{1}\NormalTok{))}
\FunctionTok{kruskal.test}\NormalTok{(ContamLevel }\SpecialCharTok{\textasciitilde{}}\NormalTok{ Predom, }\AttributeTok{data =}\NormalTok{ HAA6Br }\SpecialCharTok{\%\textgreater{}\%} \FunctionTok{group\_by}\NormalTok{(PWSID) }\SpecialCharTok{\%\textgreater{}\%} \FunctionTok{slice}\NormalTok{(}\DecValTok{1}\NormalTok{))}
\FunctionTok{kruskal.test}\NormalTok{(ContamLevel }\SpecialCharTok{\textasciitilde{}}\NormalTok{ Predom, }\AttributeTok{data =}\NormalTok{ HAA9 }\SpecialCharTok{\%\textgreater{}\%} \FunctionTok{group\_by}\NormalTok{(PWSID) }\SpecialCharTok{\%\textgreater{}\%} \FunctionTok{slice}\NormalTok{(}\DecValTok{1}\NormalTok{))}
\end{Highlighting}
\end{Shaded}

\hypertarget{ethnicity-tests}{%
\subsection{Ethnicity tests}\label{ethnicity-tests}}

\hypertarget{ethnicity-kruskal-wallis}{%
\subsubsection{Ethnicity
Kruskal-Wallis}\label{ethnicity-kruskal-wallis}}

\begin{Shaded}
\begin{Highlighting}[]
\FunctionTok{ggplot}\NormalTok{(full, }\FunctionTok{aes}\NormalTok{(}\AttributeTok{x =}\NormalTok{ HLFiftyPer, }\AttributeTok{y =}\NormalTok{ ContamLevel)) }\SpecialCharTok{+}
  \FunctionTok{geom\_violin}\NormalTok{(}\AttributeTok{color =} \StringTok{"black"}\NormalTok{, }\AttributeTok{fill =} \StringTok{"goldenrod"}\NormalTok{) }\SpecialCharTok{+}
  \FunctionTok{scale\_y\_log10}\NormalTok{() }\SpecialCharTok{+}
  \FunctionTok{facet\_wrap}\NormalTok{(}\SpecialCharTok{\textasciitilde{}}\NormalTok{ Contaminant) }\SpecialCharTok{+}
  \FunctionTok{theme\_classic}\NormalTok{() }\SpecialCharTok{+}
  \FunctionTok{labs}\NormalTok{(}\AttributeTok{title =} \StringTok{"Contaminant concentrations by Hispanic/Latin(o/a) percentile"}\NormalTok{,}
       \AttributeTok{subtitle =} \StringTok{"Note: y{-}axis is log{-}adjusted"}\NormalTok{,}
       \AttributeTok{x =} \StringTok{"Hispanic/Latin(o/a) proportion relative to 50th percentile"}\NormalTok{,}
       \AttributeTok{y =} \StringTok{"Contaminant level (\textbackslash{}u03bcg/L)"}\NormalTok{)}
\end{Highlighting}
\end{Shaded}

\begin{Shaded}
\begin{Highlighting}[]
\FunctionTok{kruskal.test}\NormalTok{(ContamLevel }\SpecialCharTok{\textasciitilde{}}\NormalTok{ HLFiftyPer, }\AttributeTok{data =}\NormalTok{ germanium }\SpecialCharTok{\%\textgreater{}\%} \FunctionTok{group\_by}\NormalTok{(PWSID) }\SpecialCharTok{\%\textgreater{}\%} \FunctionTok{slice}\NormalTok{(}\DecValTok{1}\NormalTok{))}
\FunctionTok{kruskal.test}\NormalTok{(ContamLevel }\SpecialCharTok{\textasciitilde{}}\NormalTok{ HLFiftyPer, }\AttributeTok{data =}\NormalTok{ manganese }\SpecialCharTok{\%\textgreater{}\%} \FunctionTok{group\_by}\NormalTok{(PWSID) }\SpecialCharTok{\%\textgreater{}\%} \FunctionTok{slice}\NormalTok{(}\DecValTok{1}\NormalTok{))}
\FunctionTok{kruskal.test}\NormalTok{(ContamLevel }\SpecialCharTok{\textasciitilde{}}\NormalTok{ HLFiftyPer, }\AttributeTok{data =}\NormalTok{ HAA6Br }\SpecialCharTok{\%\textgreater{}\%} \FunctionTok{group\_by}\NormalTok{(PWSID) }\SpecialCharTok{\%\textgreater{}\%} \FunctionTok{slice}\NormalTok{(}\DecValTok{1}\NormalTok{))}
\FunctionTok{kruskal.test}\NormalTok{(ContamLevel }\SpecialCharTok{\textasciitilde{}}\NormalTok{ HLFiftyPer, }\AttributeTok{data =}\NormalTok{ HAA9 }\SpecialCharTok{\%\textgreater{}\%} \FunctionTok{group\_by}\NormalTok{(PWSID) }\SpecialCharTok{\%\textgreater{}\%} \FunctionTok{slice}\NormalTok{(}\DecValTok{1}\NormalTok{))}
\end{Highlighting}
\end{Shaded}

\hypertarget{ethnicity-linear-regressions}{%
\subsubsection{Ethnicity linear
regressions}\label{ethnicity-linear-regressions}}

\begin{Shaded}
\begin{Highlighting}[]
\FunctionTok{ggplot}\NormalTok{(full, }\FunctionTok{aes}\NormalTok{(}\AttributeTok{x =}\NormalTok{ HispLat\_prop, }\AttributeTok{y =}\NormalTok{ LogLevel)) }\SpecialCharTok{+}
  \FunctionTok{geom\_point}\NormalTok{(}\AttributeTok{alpha =}\NormalTok{ .}\DecValTok{1}\NormalTok{) }\SpecialCharTok{+}
  \FunctionTok{geom\_smooth}\NormalTok{(}\AttributeTok{se =} \ConstantTok{FALSE}\NormalTok{, }\AttributeTok{method =} \StringTok{"lm"}\NormalTok{, }\AttributeTok{color =} \StringTok{"goldenrod"}\NormalTok{) }\SpecialCharTok{+}
  \FunctionTok{theme\_classic}\NormalTok{() }\SpecialCharTok{+}
  \FunctionTok{facet\_wrap}\NormalTok{(}\SpecialCharTok{\textasciitilde{}}\NormalTok{ Contaminant) }\SpecialCharTok{+}
  \FunctionTok{scale\_x\_continuous}\NormalTok{(}\AttributeTok{labels =}\NormalTok{ scales}\SpecialCharTok{::}\NormalTok{percent) }\SpecialCharTok{+}
  \FunctionTok{labs}\NormalTok{(}\AttributeTok{title =} \StringTok{"Contaminant concentrations by proportion of Hispanic/Latin(o/a)"}\NormalTok{,}
       \AttributeTok{x =} \StringTok{"Proportion of Hispanic/Latin(o/a) constituents"}\NormalTok{,}
       \AttributeTok{y =} \StringTok{"Concentration (log{-}transformed \textbackslash{}u03bcg/L)"}\NormalTok{)}

\NormalTok{germanium\_model\_eth }\OtherTok{\textless{}{-}} \FunctionTok{glm}\NormalTok{(LogLevel }\SpecialCharTok{\textasciitilde{}}\NormalTok{ HispLat\_prop, }\AttributeTok{data =}\NormalTok{ germanium)}
\FunctionTok{summary}\NormalTok{(germanium\_model\_eth)}

\NormalTok{manganese\_model\_eth }\OtherTok{\textless{}{-}} \FunctionTok{glm}\NormalTok{(LogLevel }\SpecialCharTok{\textasciitilde{}}\NormalTok{ HispLat\_prop, }\AttributeTok{data =}\NormalTok{ manganese)}
\FunctionTok{summary}\NormalTok{(manganese\_model\_eth)}

\NormalTok{HAA6Br\_model\_eth }\OtherTok{\textless{}{-}} \FunctionTok{glm}\NormalTok{(LogLevel }\SpecialCharTok{\textasciitilde{}}\NormalTok{ HispLat\_prop, }\AttributeTok{data =}\NormalTok{ HAA6Br)}
\FunctionTok{summary}\NormalTok{(HAA6Br\_model\_eth)}

\NormalTok{HAA9\_model\_eth }\OtherTok{\textless{}{-}} \FunctionTok{glm}\NormalTok{(LogLevel }\SpecialCharTok{\textasciitilde{}}\NormalTok{ HispLat\_prop, }\AttributeTok{data =}\NormalTok{ HAA9)}
\FunctionTok{summary}\NormalTok{(HAA9\_model\_eth)}
\end{Highlighting}
\end{Shaded}




\end{document}
